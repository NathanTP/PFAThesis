The problem of deep and heterogeneous memory hierarchies is not entirely new;
previous mainframe and high-performance computing platforms have previously
exposed the concept of remote memory. I will describe some of these approaches
in the next two sections.

\subsubsection{Low-Level Interfaces}
\paragraph{NUMA}
\Gls{numa} architectures partition memory resources across several compute
nodes such that memory is always local to exactly one compute resource, but
still directly addressable by the others. In this case, all memory has the same
interface (loads and stores from CPUs), but some is faster than others
(non-uniform). Some \gls{numa} systems include hardware services to aid in page
migration to mitigate this effect\cite{sgi_origin}.

\Gls{numa} systems are appealing because they appear to software as a single,
large memory. They can also offer memory access latencies on the order of 100s
of nanoseconds. This performance and tight coupling, however, limit
scalability. \Gls{numa} the largest NUMA systems can scale to hundreds of nodes
and 10s of TB of memory\cite{sgiUV}, but typical systems support only a few TB
and less than 10 nodes (due to poor scaling in cost and power).

\paragraph{RDMA}
\gls{rdma} systems are similar to NUMA in that memory resources are partitioned
among several compute nodes (memory is always local to
someone)\cite{RoCE}\cite{RFC5040}. The difference is that while NUMA systems
typically expose a cache-coherent load-store interface to both local and remote
memory resources, \gls{rdma} uses a special put/get interface to access remote
memory resources.  Typically, this service is provided through the network
interface and managed by software. This interface allows RDMA systems to scale
beyond what is possible in NUMA systems, at the cost of remote memory access
performance and a more complex interface to applications.

\Gls{rdma} systems can scale to thousands of nodes and petabytes of
memory\cite{IB_ref_design}. Performance can vary, and scales with deployment
size, but modern Infiniband networks provide round-trip latencies of several
microseconds (within a rack) and bandwidths of hundreds of gigabytes per
second\cite{ib_perf}. These systems have historically been considered costly
and only were primarily deployed in supercomputing environments, but recent
Ethernet-based implementations have made them increasingly
accessible\cite{RoCE}.
 
\paragraph{Memory Semantic Fabrics}
Finally, a new class of interface has been recently introduced; the
memory-semantic fabric. A memory-semantic fabric abstracts memory into a simple
load-store interface (rather than technology-specific protocols). These interfaces are tightly coupled with the CPU, often loading memory directly into local caches or even registers. This
abstraction enables heterogeneous memory technologies in flexible topologies.
Memory thus becomes a first-class citizen (often called a "memory blade") on a
memory-optimized interconnect. The hope is that such interfaces will allow for
greater scalability and flexibility than NUMA, while providing a more direct
interface than RDMA. There are several commercial consortia developing
cache-coherent interconnects for integrating accelerators and memories within a
rack \cite{ccix}\cite{capi}. Some academic projects have focused on scaling
NUMA by increasing the level of abstraction (e.g.,
\cite{sonuma}\cite{lim_disag}).  Finally, an industrial effort called Gen-Z
provides a more general interface that can connect memory, accelerators, and
storage using memory-oriented operations (like load and store)\cite{genz}.
While Gen-Z does not include cache-coherence in the core specification, it can
be added through custom commands between devices that require it. It remains to
be seen how these new interconnects balance performance, scalability, and cost.

\subsubsection{Software Interfaces} The low level interfaces listed above do
not necessarily mandate a particular software interface. NUMA systems typically
expose a virtual memory abstraction to applications. In this case, the OS
manages mappings from virtual to physical addresses while hardware uses those
mappings to automatically route loads and stores to the appropriate memory
resources. The OS is also responsible for choosing which NUMA domain to
allocate memory from. This can be a complex decision and much effort has gone
into studying such allocation policies\cite{linux_numa}.

RDMA systems are further divorced from specific hardware interfaces and enjoy a
great diversity of interfaces. Some programming languages use a partitioned
global address space to make it appear as if language-level variables are all
directly accessible\cite{upc}\cite{grappa}. Other systems use RDMA more
directly to accelerate applications such as key-value
stores\cite{ramcloud}\cite{farm}.

Memory-semantic fabrics are newer and it is not clear how their interfaces
should be exposed. By coupling tightly with CPUs, it is possible to address
them directly using virtual memory. However, it may be desirable to allow
applications to choose which memory they access, or have more abstracted
interfaces (e.g., disk-like). Furthermore, these fabrics are designed to
support highly heterogeneous memory technologies. This has lead to
page-migration proposals that try to manage performance and durability
requirements either explicitly in the application, or transparently in the
OS\cite{heteros}\cite{mojim}.

In this thesis, I will focus on a very general interface called demand paging
(covered in detail in the next section) that can be implemented under any of
the low-level interfaces listed here. We assume a system that allows block
reads and writes to remote memory resources. In a \gls{numa} system, this would
translate to page migration. For \gls{rdma}, we would allocate memory from
under-utilized nodes to store pages from oversubscribed nodes (as was done in
\cite{infiniswap}). In the memory-semantic approach, dedicated memory blades
would be used for remote memory and transfers would be initiated directly from
the client CPUs (e.g., using \emph{memcpy()}).
