Disaggregated memory systems promise to simplify deployment, allocation, and
scheduling on next-generation \glspl{wsc}, but bring significant performance
and complexity challenges. This complexity cannot be mitigated with shallow
application changes because assumptions about system performance run deep.
While caching attempts to avoid these challenges, we find that virtual memory
paging is no different; the OS machanisms implementing paging were designed in
a world with millisecond-level access latencies and are not suitible for the
microsecond access times offered by remote memory. The PFA allows us to make
the deep changes needed to accomodate this new environment. By improving
end-to-end application performance by up to 40\%, the PFA enables a greater
range of applications to run on limited local memory. However, caching is a
general purpose approach, and applications with large working sets and poor
locality will always suffer from increased main memory access times. To take
full advantage of disaggregated memory, we will need a mix of interfaces, both
implicit, and explicit.

