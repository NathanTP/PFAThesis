% For stuff that's both an acryonym and a glossary entry
\newcommand*{\newdualentry}[5][]{%  
  \newglossaryentry{main-#2}{name={#4},%  
  text={#3\glsadd{#2}},%  
  description={#5},%  
  #1  
  }%  
  \newacronym{#2}{#3\glsadd{main-#2}}{#4}%  
}

% \glsdisablehyper
\renewcommand*{\glstextformat}[1]{\textcolor{black}{#1}}

\newdualentry{wsc}
  {WSC}
  {warehouse-scale computer}
  {Generic term referring to tightly-integrated clusters of
    machines deployed in the datacenter.}

\newdualentry{sip}
  {SiP}
  {system in package}
  {A system where all the necessary components (NIC, cpu, memory)
   are grouped into the same physical package (but not necessarily on the same
   chip)}

\newglossaryentry{paging}
{
  name={paging},
  description={The process of storing logical pages on an external storage
device in order to free physical memory. Also called "swapping".}
}

\newdualentry{pfa}
  {PFA}
  {page fault accelerator}
  {The proposed hardware-accelerator that handles page-faults for
  remote pages automatically.}

\newglossaryentry{disag}
{
  name={dissaggregation},
  description={The WSC design that moves compute resources (such as memory,
disk, and CPUs) into dedicated resource-blades that are connected through a
high-performance network.}
}

\newdualentry{numa}
  {NUMA}
  {non-uniform memory access}
  {A system where memory is cache-coherently available to multiple CPUS, but
   with varying access latencies and bandwidths (a type of multi-socket
   machine).}

\newdualentry{rdma}
  {RDMA}
  {remote direct memory access}
  {A system where memory is directly addressable between multiple
   nodes through a network interface. RDMA systems are not typically
   cache-coherent.}

\newglossaryentry{page}
{
  name={page},
  description={A fixed-size logical group of data (typically
\SI{4}{\kibi\byte}). Sometimes called ``virtual page''.}
}

\newglossaryentry{page frame}
{
  name={page frame},
  description={A fixed-size region of physical memory used to store pages.}
}

\newglossaryentry{frame}
{
  name={frame},
  description={Synonym for page frame},
  see=[Glossary:]{page frame}
}

\newglossaryentry{pgtbl}
{
  name={page table},
  description={A hardware-visible tree in main memory that contains
translations from virtual to physical addresses.}
}

\newdualentry{pte}
  {PTE}
  {page table entry}
  {A single entry of the page-table. Each PTE refers to a single
  virtual page.}

\newdualentry{tlb}
  {TLB}
  {translation look-aside buffer}
  {A cache of virtual to physical address translations.}

\newdualentry{ptw}
  {PTW}
  {page table walker}
  {A hardware device to automatically walk the page-table and
  locate PTEs for a particular virtual address.}

\newglossaryentry{bookkeeping}
{
  name={bookkeeping},
  description={The internal OS-specific tasks related to bringing in a new
page. This typically includes updating meta-data and other page-tracking
activities.}
}

\newglossaryentry{freeq}
{
  name={FreeQ},
  description={Queue of free frames to be used by the PFA to service
page-faults.}
}

\newglossaryentry{newq}
{
  name={NewQ},
  description={Queue of new-page descriptors populated by the PFA on every page
fault and drained by the OS for bookkeeping.}
}

\newglossaryentry{evictq}
{
  name={EvictQ},
  description={Queue of pages to be evicted by the PFA. Populated by the OS
when it needs to free local physical memory.}
}

\newglossaryentry{pgid}
{
  name={pageID},
  description={A unique identifier for a page in remote memory. Acts as a
remote-memory address.}
}

\newglossaryentry{memory blade}
{
  name={memory blade},
  description={A dedicated memory node in a disaggregated system. The memory blade
exists solely to server memory requests. A memory blade may be custom-designed
for this purpose, or may simply expose an RDMA interface.}
}

\newdualentry{mtu}
  {MTU}
  {maximum transfer unit}
  {The largest contiguous packet that a network is capable of
  transmitting.}

\newdualentry{tmem}
  {TMem}
  {transcendent memory}
  {A layer in the Linux paging subsystem that stores pages in
  specialized memory that may not be disk-backed.}

\newglossaryentry{cgroup}
{
  name={cgroup},
  first={control group (cgroup)},
  description={The per-task (or group of tasks) resource management system in
the Linux kernel.}
}

\newglossaryentry{swap}
{
  name={swap},
  description={Historically used to refer to the process of moving an entire
process's memory image to disk, Linux uses ``swapping'' to refer to all
paging.}
  see=[Glossary:]{paging}
}

\newglossaryentry{anonpg}
{
  name={anonymous page},
  description={A page that does not contain disk-backed information. This is
primarily ``heap'' memory (e.g. memory allocated through malloc())}
}

\newdualentry{vma}
  {VMA}
  {virtual memory area}
  {Contiguous region of virtual memory used by Linux to simplify memory
   management.}

\newglossaryentry{swpent}
{
  name={swap\_entry\_t},
  description={A Linux-specific value stored in evicted PTEs that contains
information on where to locate an evicted page.}
}

\newglossaryentry{kpfad}
{
  name={kpfad},
  description={A background daemon that opportunistically performs bookkeeping
  and maintenance for the PFA.}
}

\newglossaryentry{kswapd}
{
  name={kswapd},
  description={A background kernel thread that opportunistically performs
    bookkeeping.}
}

\newglossaryentry{task}
{
  name={task},
  description={Linux kernel internal abstraction of a process.}
}

\newdualentry{lru}
  {LRU}
  {least recently used}
  {An algorithm that attempt to pick pages that have not been used recently.}

\newdualentry{nvm}
  {NVM}
  {non-volatile memory}
  {Storage devices with near-DRAM performance, and byte-addressiblity, that do
  not lose their data when powered off.}

\newdualentry{tco}
  {TCO}
  {total cost of ownership}
  {A metric that includes not only up-front costs of a system, but the total
  cost to own and operating that system for its effective lifespan.}
