The PFA was implemented within the RISC-V ecosystem. RISC-V is an open-source
instruction set with several open and closed-source implementations and ports
for many common software components\cite{riscv}. I used the RISC-V port of
Linux 4.13 with a Buildroot generated user
space.

\subsection{PFA Reference Implementation}
In order to accelerate software development, and to provide a golden-model of
PFA behavior, I implemented the PFA first in a RISC-V ISA simulator called
"Spike"\cite{spike}. Spike provides a functional simulation of a RISC-V core
through a straightforward C++ interpreter, but does not provide any timing
accuracy. Due to its simplicity, the PFA implementation required only a few
weeks of implementation effort and less than 1000 LoC. With Spike, software
development was able to proceed concurrently with the concrete hardware design.
Furthermore, unit tests developed under Spike were used to validate the
hardware implementation, reducing debugging effort. In all, the only software
change that was needed to go from Spike to a concrete implementation was one
extra TLB flush due to a difference in TLB design between Spike and the RISC-V
implementation we used.

\subsection{Hardware Implementation}
The PFA was implemented in the Chisel hardware construction
language\cite{chisel} and integrated with a simple in-order CPU called
RocketCore\cite{rocketCore}. The components were integrated using the
RocketChip system-on-chip (SoC) generator\cite{rocketChip}. I provide an
overview of the relevant systems in the following sections. The current PFA
prototype implements a subset of the specification described in
section \ref{sec:pfaDesign}. Specifically, it does not support multiple
simultaneous evictions (the EvictQ has an effective depth of 1), however, it
does allow for asynchronous eviction. The current prototype also does not
support the full memory blade specification. Instead, it sends a simplified form
of packet header with only opcode and transaction ID. It also does not use a
write acknowledgment (the OS must rate-limit writes to avoid overwhelming the
memory blade).
 
\subsubsection{RocketCore and RocketChip}
RocketChip\cite{rocketChip} is a framework for generating SoCs. It includes on-chip
interconnects, caches, and other utilities for chip construction. While the CPU
is pluggable, we use only a single RocketCore in-order CPU for our experiments. Our
implementation used dedicated 16KB instruction and data caches\todo{I hope to
run experiments with an L2 in the future}. While the node had access to several
gigabytes of main memory, we artificially limited application memory using
Linux cgroups (see section \ref{sec:eval} for details).

\subsubsection{Memory Blade}
The memory blade that is used by the PFA was implemented as a bare-metal C
program running on a RocketCore-based compute node. This memory blade is
connected to the same network as our PFA-enabled compute node and traverses a
simple switch in order to communicate. The current implementation uses a simple
protocol in lieu of a full ethernet stack. 

\subsubsection{FireSim}
We simulated the RTL using a cycle-accurate simulator called
"FireSim"\cite{firesim}. FireSim is an FPGA-accelerated simulator that runs on
the Amazon cloud. It can simulate thousands of nodes with a cycle-accurate
network and heterogenous components.  Many parameters of the simulation are
tuneable within FireSim. We decided on a 200Gbps network with 2us link latency,
leading to roughly 8us page access time to the memory blade. The network is
in-order and reliable. \todo{figure out DRAM parameters}

\subsection{Linux Integration} \label{sec:linuxImpl}
    \todo[inline]{Write Linux Implementation section}
This section covers the bulk of my personal implementation efforts. It will cover how Linux works right now internally, and then the changes that were made to integrate PFA support. It will also go into the bookkeeping daemon (kpfad) and some of the design considerations there.
