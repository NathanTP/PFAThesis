\begin{abstract}

There have been several recent proposals to disaggregate memory in
warehouse-scale computers, motivated by the increasing performance of networks,
and a proliferation of novel memory technologies (e.g. HBM, NVM). In a system
with memory disaggregation, each compute node contains a modest amount of fast
memory (e.g. high-bandwidth DRAM integrated on-package), while large capacity
memory or NVM is made available across the network through dedicated memory
nodes. One common proposal to harness the fast local memory is to use it as a
large cache for the remote bulk memory. This cache could be implemented purely
in hardware, which could minimize latency, but may involve complicated
architectural changes and would lack OS insights into memory usage.  An
alternative is to manage the cache purely in software with traditional paging
mechanisms. This approach requires no additional hardware, can use
sophisticated algorithms, and has insight into memory usage patterns. However,
our experiments show that even when paging to local memory, applications can be
slowed significantly due to the overhead of handling page faults, which can
take several microseconds and pollute the caches. In this thesis, I propose a
hybrid HW/SW cache using a new hardware device called the ``page fault
accelerator'' (PFA), with a special focus on the impact on operating system
design and performance.

\end{abstract}

